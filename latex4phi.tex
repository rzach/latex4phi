
\usepackage{hyperref}

\newenvironment{articleonly}{\mode<article>}{\mode<all>}

\mode<presentation>{\usetheme{Frankfurt}}

\title{\LaTeX{} for Philosophers}

\author{Richard Zach}

\begin{document}

\frame{\maketitle}

  \frame{
\tableofcontents[ 
    currentsubsection, 
    hideothersubsections, 
    sectionstyle=show/hide, 
    subsectionstyle=show/shaded, 
    ] }

\section{What is \LaTeX?}

\begin{frame}
  \frametitle{\LaTeX{} is a Typesetting System}

\begin{itemize}
\item \LaTeX{} is a progam that takes a plain text file containing
  text and markup instructions and turns it into a typeset document
  (usually in PDF format).
\item \LaTeX{} is open source and cross-platform.
\item \LaTeX{} comes with hundreds of \emph{packages} dealing with
  anything from producing presentation slides (such as this) to
  dealing with bibliographical references to typesetting musical
  scores.
\end{itemize}
\end{frame}

\begin{frame}[fragile]
\frametitle{\LaTeX{} is a Markup Language}

\begin{itemize}
\item \LaTeX{} documents do not contain formatting---all formatting,
  sectioning, special symbols and formulas, etc., are coded using
  plain text.
\item For instance:
  \begin{itemize}
    \item an \emph{emphasized} piece of text is coded as
      \verb+\emph{emphasized}+,
    \item a formula such as \[\exists x(\phi(x) \land \psi(x))\] as
      \begin{center}
      \verb+$\exists x(\phi(x) \land \psi(x))$+,
      \end{center}
    \item a bulleted list as 
\begin{verbatim}
\begin{itemize}
  \item \LaTeX{} documents do not contain ...
  \item ...
\end{itemize}
\end{verbatim}
  \end{itemize}
\end{itemize}

\end{frame}

\begin{frame}
\frametitle{\LaTeX{} is a Programming Language}

\begin{itemize}
\item The commands of \LaTeX{} (and the underlying \TeX{} system) are
  \emph{also} a full-fledged programming language.
\item This allows you to customize your typesetting environment for
  special purposes.
\item It makes \LaTeX{} extremely flexible and powerful.
\item It also makes it very complex, and sometimes very frustrating.
\end{itemize}
\end{frame}

\begin{frame}[fragile]
\frametitle{A Simple \LaTeX{} Document}

\begin{verbatim}
\documentclass{article}

\usepackage{biblatex}
\addbibresource{latex4phi.bib}

\title{A Simple \LaTeX{} Document}
\author{Richard Zach}

\begin{document}

\maketitle

\begin{abstract}
A skeleton file illustrating the basic components of a \LaTeX{}
 document.
\end{abstract}

\section{Introduction}

\LaTeX{} was originally developed in 1985, based on the \TeX{}
 typesetting system released by Don Knuth in 1979
 \cite{Knuth1986,Lamport1986}.

\printbibliography

\end{document}

\end{verbatim}
\end{frame}


\section*{Who Should Use \LaTeX{}?}

\begin{frame}
\frametitle{Use \LaTeX{} if \dots }

\begin{itemize}
\item Your writing includes lots of formulas (logic or mathematics). 
  \begin{itemize}
  \item logicians 
  \item formal epistemologists
  \item philosophers of physics
  \item philosophers of mathematics
  \end{itemize}

\item You have to include text in other languages or alphabets, code,
or diagrams.
\begin{itemize}
\item philosophers of language (phrase structure trees)
\item ancient philosophers (greek text with diacritics)
\end{itemize}

\end{itemize}

\end{frame}

\begin{frame}
\frametitle{Use \LaTeX{} if \dots}

\begin{itemize}
\item You want no control (and hence not have to worry about) the
  layout of your writing.

\begin{articleonly}
If you use \LaTeX{} and stay away from formatting commands, your
typeset output will basically be guaranteed to look good.  You can
change the look of your paper very easily and without having to format
every single paragraph.  This can be done in Word, but is much more
cumbersome than in \LaTeX.
\end{articleonly}

\item You want complete control over the layout of your writing and
  produce professionally typeset output.

\begin{articleonly}
On the other hand, if you want to typeset your writing professionally,
\LaTeX{} is about as capable as professional typesetting
software. Packages like \cite{memoir} are designed to produce
print-ready books according to traditional publishing standards. This
is hard to do in Word.
\end{articleonly}

\item You are fussy about typography.

\begin{articleonly}
Word processors such Word have come a long way in turning text into
type on the page---in part because of the sophisticated algorithms
that underly the \TeX{} system.  \LaTeX{} understands kerning, line
spacing, hyphenation, its typefaces have ligatures, its \textsc{small
  caps} aren't just scaled-down capitals, and it makes it easy to keep
your hyphens, en, and em-dashes straight.
\end{articleonly}
\end{itemize}

\end{frame}



\begin{frame}
\frametitle{Use \LaTeX{} if \dots}

\begin{itemize}
\item You have financial or ideological objections against commercial
  software.

\begin{articleonly}
\LaTeX{} and basically everything around \LaTeX{} is free because it's
all open source. To do the same kinds of things you can do with
\LaTeX{} for free, you'd need Word, PowerPoint (or Keynote), EndNote,
Acrobat Pro, and maybe InDesign.
\end{articleonly}

\item You want your work to be future-proof.

\begin{articleonly}
\LaTeX{} documents are plain text documents. Proprietary document
formats change and become obsolete.
\end{articleonly}

\item You want your work to be available consistently across platforms.

\begin{articleonly}
A \LaTeX{} document compiles the same on a Windows, Mac, or Unix
machine.  A Word file may lose fonts, formatting, page breaks if you
open it on a Mac or with a program other then the very version of Word
you generated it on.  A PDF presentation generated by
\LaTeX{}willdisplay the same on any computer, but you might not even
be able to open a Keynote presentation.
\end{articleonly}
\end{itemize}

\end{frame}

\begin{frame}[fragile]
\frametitle{Use \LaTeX{} if \dots}

\begin{itemize}
\item You want to stop worrying about bibliographies.

\begin{articleonly}
\LaTeX{} comes with sophisticated bibliography handling. You can keep
your bibliography in a separate Bib\TeX{} file (which is also a simple
text file), refer to bibliography entries in your text (e.g., as
\verb+\cite{Quine1951}+ and \LaTeX{} will not only print ``Quine
(1951)'' in the text but also include the correct entry in your
bibliography, formatted according to your preferred format.
\end{articleonly}

\item You want reliable tables of contents, crossreferences, and maybe
  even indexes.

\begin{articleonly}
Because a \LaTeX{} is logically marked up (e.g., using \verb+section+
  commands) and not visually formatted directly, it will alwaysbe
  clear what's a chapter heading and what's a section. Producing a
  table of contents is as easy as typing \verb+\tableofcontents+.
\end{articleonly}

\item You get easily sidetracked by techy things.

\begin{articleonly}
If you're the kind of person who has to understand the inner workings
of things, is tempted to take things apart, improve them, or repurpose
them---then be prepared to spend a lot of time tinkering with
\LaTeX. You may find yourself writing your own package files, or
wasting weeks on making your documents be just a little bit more
beautiful. This is not a good idea if you have a dissertation to
finish.
\end{articleonly}
\end{itemize}

\end{frame}

\section{Getting \LaTeX}

\begin{frame}
\frametitle{Installing the \LaTeX{} System}

\begin{itemize}
\item For Mac, get \href{http://www.tug.org/mactex/}{MacTex}
\item For everything else, get \href{http://www.tug.org/texlive/}{TeX Live}
\end{itemize}

\end{frame}

\begin{frame}
\frametitle{Useful Tools: Editors}

\begin{itemize}
\item \href{http://pages.uoregon.edu/koch/texshop/texshop.html}{TeXShop} seems to be the most widely used editor for Mac (included in MacTeX).
\item \href{http://www.tug.org/texworks/}{TeXworks} is an editor modelled after TeXShop but available for Mac, Windows, and Linux (included in MacTeX and TeXLive for Windows)
\end{itemize}

\end{frame}

\begin{frame}
\frametitle{Useful Tools: Bibliography Management}

\begin{itemize}
\item \href{http://bibdesk.sourceforge.net/}{BibDesk} is a bibliography manager which will even keep track of the corresponding PDFs for you (Mac only, included in MacTeX)
\item \href{http://jabref.sourceforge.net/}{JabRef} is cross-platform, so works on Windows and Linux aswell.
\item Other \href{http://en.wikipedia.org/wiki/Comparison_of_reference_management_software}{reference managers} like \href{http://www.citeulike.org/}{CiteULike}, \href{http://www.mendeley.com/}{Mendeley}, and \href{http://www.zotero.org/}{Zotero} can import and export Bib\TeX{} files. 
\end{itemize}

\end{frame}

\begin{frame}
\frametitle{}

\begin{itemize}
\end{itemize}

\end{frame}

\begin{frame}
\frametitle{}

\begin{itemize}
\end{itemize}

\end{frame}


\section{Packages You Should Look At}

\begin{frame}
\frametitle{}

\begin{itemize}
\end{itemize}

\end{frame}

\begin{frame}
\frametitle{}

\begin{itemize}
\end{itemize}

\end{frame}

\begin{frame}
\frametitle{}

\begin{itemize}
\end{itemize}

\end{frame}

\begin{frame}
\frametitle{}

\begin{itemize}
\end{itemize}

\end{frame}



\end{document}
