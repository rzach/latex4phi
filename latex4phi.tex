% LaTeX for Philosophers
% Richard Zach
% http://github.com/rzach/latex4phi

% This file is included by latex4phi-beamer and latex4phi-article and
% will not compile by itself

\usepackage[colorlinks,allcolors=blue]{hyperref}
\usepackage{graphicx}
  \beamersetuncovermixins
  {\opaqueness<1>{50}\opaqueness<2>{30}\opaqueness<3>{15}\opaqueness<4->{5}}%
  {\opaqueness<1>{50}\opaqueness<2>{30}\opaqueness<3>{15}\opaqueness<4->{5}}

\newenvironment{articleonly}{\mode<article>}{\mode<all>}

\mode<presentation>{\usetheme{Frankfurt}}

\title{\LaTeX{} for Philosophers}

\author{Richard Zach}

\begin{document}

\frame{\maketitle}

\mode<presentation>{
\begin{frame}{Outline}
\tableofcontents
\end{frame}}

\section{What is \LaTeX?}
\mode<presentation>{\subsection{}}

\begin{frame}{\LaTeX{} is a Typesetting System}

\begin{itemize}
\item \LaTeX{} is a program that takes a plain text file containing
  text and markup instructions and turns it into a typeset document
  (usually in PDF format).
\item (Actually, \TeX{} does the typesetting, and \LaTeX{} works through \TeX.)
\item \LaTeX{} is open source, cross-platform, and almost 30 years old.
\item \LaTeX{} comes with hundreds of \emph{packages} dealing with
  anything from producing presentation slides (such as this) to
  dealing with bibliographical references to typesetting musical
  scores.
\end{itemize}
\end{frame}

\begin{frame}[fragile]{\LaTeX{} is a Markup Language}

\begin{itemize}
\item \LaTeX{} documents do not contain formatting---all formatting,
  sectioning, special symbols and formulas, etc., are coded using
  plain text.
\item For instance:
  \begin{itemize}
    \item an \emph{emphasized} piece of text is coded as
      \verb+\emph{emphasized}+,
    \item a formula such as \[\exists x(\phi(x) \land \psi(x))\] as
      \begin{center}
      \verb+$\exists x(\phi(x) \land \psi(x))$+,
      \end{center}
    \item a bulleted list as 
\begin{verbatim}
\begin{itemize}
  \item \LaTeX{} documents do not contain ...
  \item ...
\end{itemize}
\end{verbatim}
  \end{itemize}
\end{itemize}

\end{frame}

\begin{frame}{\LaTeX{} is a Programming Language}

\begin{itemize}
\item The commands of \LaTeX{} (and the underlying \TeX{} system) are
  \emph{also} a full-fledged programming language.
\item This allows you to customize your typesetting environment for
  special purposes.
\item It makes \LaTeX{} extremely flexible and powerful.
\item It also makes it very complex, and sometimes very frustrating.
\end{itemize}
\end{frame}

\begin{frame}[fragile,allowframebreaks]{A Simple \LaTeX{} Document}

\begin{verbatim}
\documentclass[11pt]{article}

\usepackage[citestyle=authoryear,style=authoryear]{biblatex}
\addbibresource{latex4phi.bib}
\usepackage[greek,english]{babel}

\title{A Simple \LaTeX{} Document}
\author{Richard Zach}

\begin{document}

\maketitle

\section{Introduction}

\LaTeX{} was originally developed in 1985, based on the 
\TeX{} typesetting system released by Don Knuth in 1979
\parencite{Lamport1986,Knuth1986}.  You can \emph{easily} 
set  formulas like $\exists x(\phi(x) \land \psi(x))$, or 
greek text, e.g., when discussing Aristotle's 
\textgreek{>'apeiron}.

\printbibliography

\end{document}
\end{verbatim}
\end{frame}

\begin{frame}

\includegraphics[height=15cm,clip=true,trim=4cm 5cm 4cm 5cm]{latex4phi-sample}
\end{frame}

\begin{frame}[fragile]{Bib\TeX{} Bibliography Files}

\begin{itemize}
\item Bibiographic references can be generated and formatted automatically.
\item Format of Bib\TeX{} databases is simple and support is widespread.
\begin{verbatim}
@article{Thomson1971,
    author = {Thomson, Judith Jarvis},
    title = {A Defense of Abortion},
    journal = {Philosophy \& Public Affairs},
    pages = {47--66},
    volume = {1},
    year = {1971}}
\end{verbatim}
\end{itemize}
\end{frame}

\section{Who Should Use \LaTeX{}?}
\mode<presentation>{\subsection{}}

\mode<presentation>{
\begin{frame}{You Are Here}
\tableofcontents[ 
    currentsection, 
    hideothersubsections, 
    sectionstyle=show/shaded, 
    ]
\end{frame}}

\begin{frame}{Use \LaTeX{} if \dots }

\begin{itemize}
\item Your writing includes lots of formulas (logic or mathematics). 
  \begin{itemize}
  \item logicians ,
  \item formal epistemologists,
  \item philosophers of physics,
  \item philosophers of mathematics.
  \end{itemize}

\item You have to include text in other languages or alphabets, code,
or diagrams, or you need sophisticated typesetting 
\begin{itemize}
\item philosophers of language (phrase structure trees),
\item ancient philosophers (greek text with diacritics),
\item historians (for scholarly apparatus).
\end{itemize}

\end{itemize}

\end{frame}

\begin{frame}[fragile]{Use \LaTeX{} if \dots}

\begin{itemize}
\item You want no control (and hence not have to worry about) the
  layout of your writing.

\begin{articleonly}
If you use \LaTeX{}, your typeset output will basically be guaranteed
to look good.  You can change the look of your paper very easily and
without having to format every single paragraph.  This can be done in
Word, but is much more cumbersome than in \LaTeX.

There are very many styles that make sure your typeset document has
the right appearance: type size, margins, numbering of headers,
etc. This can make it very easy to prepare certain kinds of
documents---e.g., manuscripts for journals or presses that have
a \LaTeX{} house style, or a dissertation (most universities have
a \LaTeX{} thesis style or class).
\end{articleonly}

\item You want complete control over the layout of your writing and
  produce professionally typeset output.

\begin{articleonly}
On the other hand, if you want to typeset your writing professionally,
\LaTeX{} is about as capable as professional typesetting
software. Packages like \texttt{memoir} are designed to produce
print-ready books according to traditional publishing standards. This
is hard to do in Word.
\end{articleonly}

\item You want to stop worrying about bibliographies.

\begin{articleonly}
\LaTeX{} comes with sophisticated bibliography handling. You can keep
your bibliography in a separate Bib\TeX{} file (which is also a simple
text file), refer to bibliography entries in your text (e.g., as
\verb+\cite{Quine1951}+ and \LaTeX{} will not only print ``Quine
(1951)'' in the text but also include the correct entry in your
bibliography, formatted according to your preferred format.
\end{articleonly}

\item You want reliable tables of contents, cross-references, and maybe
  even indexes.

\begin{articleonly}
Because a \LaTeX{} is logically marked up (e.g., using \verb+section+
commands) and not visually formatted directly, it will always be clear
what's a chapter heading and what's a section. Producing a table of
contents is as easy as typing \texttt{\textbackslash tableofcontents}.
\LaTeX{} has sophisticated support for ``floats,'' such as figures and
tables, which it will place automatically in such a way that the rest
of the text appears well-balanced on the page. Of course it will also
automatically generate lists of tables and figures.
\end{articleonly}

\item You are fussy about typography.

\begin{articleonly}
Word processors such Word have come a long way in turning text into
type on the page---in part because of the sophisticated algorithms
that underly the \TeX{} system.  \LaTeX{} understands kerning, line
spacing, hyphenation, its typefaces have ligatures, its \textsc{small
caps} aren't just scaled-down capitals, and it makes it easy to keep
your hyphens, en-, and em-dashes straight.
\end{articleonly}
\end{itemize}

\end{frame}

\begin{frame}{Use \LaTeX{} if \dots}

\begin{itemize}
\item You have financial or ideological objections to commercial
  software.

\begin{articleonly}
\LaTeX{} and basically everything around \LaTeX{} is free because it's
all open source. To do the same kinds of things you can do with
\LaTeX{} for free, you'd probably need Word, PowerPoint (or Keynote), EndNote,
Acrobat Pro, and maybe InDesign.
\end{articleonly}

\item You want your work to be future-proof.

\begin{articleonly}
\LaTeX{} documents are plain text documents. Proprietary document
formats change and become obsolete. If you use many supplementary
packages, your file may at some future date no longer compile, but you
will at the very least be able to recover the text in it.
\end{articleonly}

\item You want your work to be available consistently across platforms.

\begin{articleonly}
A Word file or PowerPoint presentation may lose fonts or formatting,
and page breaks might shift, if you open it on a Mac or with a program
other then the very version of Word you generated it on.  A \LaTeX{}
document compiles the same on a Windows, Mac, or Unix machine.  A PDF
presentation generated by \LaTeX{} will display the same on any
computer, but you might not even be able to open a Keynote
presentation if you don't have a Mac available.
\end{articleonly}
\end{itemize}

\end{frame}


\begin{frame}[fragile]{Don't Use \LaTeX{} if \dots}

\begin{itemize}
\item Other people who don't use \LaTeX{} have to work with your 
document, such as:
\begin{itemize}
\item Non-\LaTeX-using supervisors who want to comment on your 
electronic document;
\item Non-\LaTeX-using collaborators who want to see what their 
changes will look like;
\item {\color{red} Publishers who insist on Word files.}
\end{itemize}
\item \dots If you are committed, there are ways around these problems:
\begin{itemize}
\item Supervisors can comment on PDFs or, old-school, on paper.
\item Online editors do allow collaboration and preview of documents
  without having to install \LaTeX.
\item There are conversion programs.
\end{itemize}
\end{itemize}

\end{frame}

\begin{frame}[fragile]{Don't Use \LaTeX{} if \dots}

\begin{itemize}
\item You think it's just ridiculous that you should have to \emph{debug} a
  \emph{document} before printing it.

\begin{articleonly}
Every \LaTeX{} document is also a program, and \LaTeX{} won't give you
a shiny PDF unless your document compiles without errors.  That means
you might have to spend some time ironing out the kinks in your
document.  Often it'll just be a misspelled command or a forgotten
``\texttt{\}}.''  Sometimes it'll make you want to rip your hair out.
\end{articleonly}

\item You have better things to do than to \emph{learn} a word
  processing program.

\begin{articleonly}
\LaTeX{}, like much of open-source software written by Comp Sci types,
isn't exactly user friendly. If you don't want to read a manual, learn
commands, or shudder at the thought of using a command-line terminal
for anything (or don't know what that is!) then think twice before
installing \LaTeX---then maybe stick with the online editors or LyX.
\end{articleonly}

\item You have better things to do than spend half a
  day on the internet to figure out how to change the page margins of
  your document.

\begin{articleonly}
\LaTeX{} is very powerful and can do lots of things---but it's often
hard to figure out how to do them.  You might end up spending lots of
time searching the internet to figure things out. On the upside,
because \LaTeX{} is so widely used, it's also likely that you'll find
the answer (maybe even among your facebook friends.)
\end{articleonly}

\item You are in danger of getting sidetracked by nerdy techy things.

\begin{articleonly}
If you're the kind of person who has to understand the inner workings
of things, is tempted to take things apart, improve them, or repurpose
them---then be prepared to spend a lot of time tinkering with
\LaTeX. You may find yourself writing your own package files, or
wasting weeks on making your documents be just a little bit more
beautiful. This is not a good idea if you have a dissertation to
finish.
\end{articleonly}
\end{itemize}

\end{frame}

\section{Setting Up \LaTeX}
\mode<presentation>{\subsection{}}

\mode<presentation>{
\begin{frame}{You Are Here}
\tableofcontents[ 
    currentsection, 
    hideothersubsections, 
    sectionstyle=show/shaded, 
    ]
\end{frame}}

\begin{frame}{Getting Started with \LaTeX{} without Hassles}

\begin{itemize}
\item There are online \LaTeX{} editors, e.g.,
\begin{itemize}
\item \href{https://www.overleaf.com/}{Overleaf} and 
\item \href{https://www.sharelatex.com/}{ShareLaTeX}
\end{itemize}
Nothing to install, documents live in the cloud, share and collaborate.
\item \href{http://www.lyx.org/}{LyX} is an open-source WYSIWYG editor
  that uses \LaTeX{} and runs on Windows, Mac, and Linux.
\end{itemize}

\end{frame}

\begin{frame}{Installing the \LaTeX{} System}

\begin{itemize}
\item When you want more speed, control, a better editor, or advanced 
options, install it on your computer
\item For Mac, get \href{http://www.tug.org/mactex/}{MacTex}.
\item For everything else, get \href{http://www.tug.org/texlive/}{TeX Live}.
\end{itemize}

\end{frame}

\begin{frame}
\frametitle{Useful Tools: Editors}

\begin{itemize}
\item A good \LaTeX-aware editor is important:
\begin{itemize}
\item Helps you learn commands and save typing.
\item Useful for previewing and debugging.
\item Spell checker has to go around commands.
\end{itemize}
\item \href{http://pages.uoregon.edu/koch/texshop/texshop.html}{TeXShop}
 seems to be the most widely used editor for Mac (included in MacTeX).
\item \href{http://www.tug.org/texworks/}{TeXworks} is an editor 
modelled after TeXShop but available for Mac, Windows, and Linux 
(included in MacTeX and TeXLive for Windows).
\item There are
  \href{http://en.wikipedia.org/wiki/Comparison_of_TeX_editors}{many
    others}!
\end{itemize}

\end{frame}

\begin{frame}
\frametitle{Useful Tools: Bibliography Management}

\begin{itemize}
\item \href{http://bibdesk.sourceforge.net/}{BibDesk} is a
  bibliography manager which will even keep track of the corresponding
  PDFs for you (Mac only, included in MacTeX).
\item \href{http://jabref.sourceforge.net/}{JabRef} is cross-platform,
  so works on Windows and Linux as well.
\item Other
  \href{http://en.wikipedia.org/wiki/Comparison_of_reference_management_software}{reference
    managers} like \href{http://www.citeulike.org/}{CiteULike},
  \href{http://www.mendeley.com/}{Mendeley}, and
  \href{http://www.zotero.org/}{Zotero} can import and export
  Bib\TeX{} files.
\end{itemize}

\end{frame}

\begin{frame}{Conversion Tools}

\begin{itemize}
\item \href{http://www.sciweavers.org/convert-latex-to-rtf}{l2rtf}
  is an online tool to convert \LaTeX{} to RTF.
\item \href{http://johnmacfarlane.net/pandoc/}{pandoc} can convert to
  and from dozens of formats (and it's written by philosopher
  John MacFarlane!)---including \LaTeX.
\item \href{http://dlmf.nist.gov/LaTeXML/}{LaTeXML} is
  way for \LaTeX{} to output XML or HTML instead of PDF. 
\item
  \href{https://extensions.libreoffice.org/extensions/writer2latex-1}{Write2LaTeX}
  is an extension that lets
  \href{http://www.libreoffice.org/}{LibreOffice} export to \LaTeX{}
\item Use these and Word/LibreOffice to do a word count, or directly
  using \href{http://app.uio.no/ifi/texcount/index.html}{texcount}.
\end{itemize}

\end{frame}




\section{Packages You Should Look At}
\mode<presentation>{\subsection{}}

\mode<presentation>{
\begin{frame}{You Are Here}
\tableofcontents[ 
    currentsection, 
    hideothersubsections, 
    sectionstyle=show/shaded, 
    ]

\end{frame}}

\begin{frame}{Layout: \texttt{memoir}}

\begin{itemize}
\item \texttt{memoir} is a document class for producing anything from
  articles to books.
\item It provides ability to customize layout of your document,
  chapter and section headings, etc.
\item It includes many useful packages which relate to layout of:
\begin{itemize}
\item tables,
\item tables of contents,
\item page headers,
\item endnotes.
\end{itemize}
\end{itemize}

\end{frame}

\begin{frame}{Linking: \texttt{hyperref}}

\begin{itemize}
\item Put hyperlinks to websites in your documents
\item Automatically put in-document links from TOC to sections,
  citations to bibliography, etc.
\item Also lets you set options in the final PDF such as bookmarks,
  page mode, title, author, etc.
\end{itemize}

\end{frame}

\begin{frame}[fragile]{Bibliographies: \texttt{natbib}}

\begin{itemize}
\item The ``standard'' way to do bibliographic references
\item Supports author-year style:
\begin{itemize}
\item \verb+\citep{Quine1951}+ produces a parenthetical citation: (Quine 1951).
\item \verb+\citet{Quine1951}+ produces an in-text citation: Quine~(1951).
\end{itemize}
\item Works with Bib\TeX{} to turn bibliography databases into
  bibliographies for your document.
\item Most journals have bibliography styles that work with \texttt{natbib}.
\end{itemize}

\end{frame}

\begin{frame}{Bibliographies: \texttt{biblatex}}

\begin{itemize}
\item The modern way to do bibliographies in \LaTeX.
\item Supports not only author-year style but also in-text citations
  and footnoted references (footnotes contain bibliographic
  information, with or without separate bibliography).
\item Supports bibliography databases in the cloud.
\item Journals unlikely to support \texttt{biblatex} yet
\item Most conversion programs don't work.
\end{itemize}

\end{frame}

\begin{frame}{Presentations: \texttt{beamer}}

\begin{itemize}[<+->]
\item Produces presentations as PDFs (use full-screen mode).
\item Comes with many predefined themes.
\item Lets audience track progress through presentation.
\item Generate handout and presentation from same document.
\item Supports ``overlays'' to gradually uncover text.
\end{itemize}

\end{frame}

\section{Conclusion}
\mode<presentation>{\subsection{}}

\begin{frame}{Conclusion}

\begin{itemize}
\item It's good to know what \LaTeX{} is---even if you don't want to
  use it---and now you do.
\item You might have to or want to use it some or most of the time.
\begin{itemize}
\item Lots of people use it, even some publishers.
\item It makes really pretty documents.
\item It might give you an edge for a job (RA, editorial assistant).
\item Once you've learned to use it, it can be a big time saver.
\item If your documents are simple, using \LaTeX{} can be simple too.
\end{itemize}
\item You know the pitfalls and dangers:
\begin{itemize}
\item It's rough around the edges.
\item Don't get sucked in---your work is more important!
\end{itemize}
\end{itemize}
\end{frame}

\begin{frame}{More Reading}

\begin{itemize}
\item Everything \LaTeX{} related can be found on the
  \href{http://www.ctan.org/}{Comprehensive \TeX{} Archive Network
    CTAN}.
\item \href{http://en.wikibooks.org/wiki/LaTeX}{WikiBooks: \LaTeX}.
\item
  \href{http://individual.utoronto.ca/williecostello/latex.html}{\LaTeX{}
    for Luddites (An Ancient Philosopher's Guide)}.
\item
  \href{http://www.logicmatters.net/latex-for-logicians/}{\LaTeX{} for
    Logicians}.
\item Need a symbol in \LaTeX? Try
  \href{http://detexify.kirelabs.org/classify.html}{Detexify}.
\end{itemize}

\end{frame}

\end{document}
